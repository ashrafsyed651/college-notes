\documentclass[english,course,fleqn]{lecture}
%
% Packages
\usepackage{geometry}
\usepackage{amssymb}
\usepackage{amsmath}
\usepackage{amsthm}
\usepackage{microtype}

% Page Setting
\geometry{a4paper}

% Custom Settings
\newenvironment{qanda}{\setlength{\parindent}{0pt}}{\bigskip}
\newcommand{\Q}{\bigskip\bfseries Q:\ }
\newcommand{\A}{\par\textbf{A:} \normalfont}

% Headers
\title{Linear Transformations}
\subtitle{Unit 2}
\shorttitle{Linear Transformations}
% \subject{Subject of the Talk}
\author{Pranaov\ S}
% \email{Author's email}
\speaker{Dr.\ Venugopal}
% \spemail{Speaker's email}
\ccode{MA1001}
\date{DD}{MM}{YYYY}
% \dateend{}{}{}
% \flag{An extra line if you need it}
% \attn{Something to get reader's attention}
\morelink{https://github.com/pranaovs/college-notes}


\begin{document}

\newpage

A Transformation/Mapping/function defines a relationship between one or more variables.
It is written as $T:X\rightarrow Y$.

$X$ is said to be domain and $Y$ is said to be co-domain.
If $x \in X$ then the image is $T(x)\in Y$.

\section{Linear Transformation/Map/Operator}

\subsection{Linear Transformation}

\begin{definition}[Linear Transformation]
	Let $V$ and $W$ be vector spaces over $F$.
	A function $T:V \rightarrow W$ is called a linear transformation from $V \rightarrow W$, if:

	\begin{enumerate}
		\item $T(x+y) = T(x) + T(y)$
		\item $T(\alpha x) = \alpha T(x)$

		      (for all $x,y \in V$ and $\alpha \in F$)
	\end{enumerate}
\end{definition}

\textbf{Note:}
\begin{enumerate}
	\item These two conditions imply that $T$ preserves addition and scalar multiplication.
	\item These two conditions can be combined into a single condition to give an alternate definition:

	      Let $V$ and $W$ be vector spaces over $F$.
	      A function $T:V \rightarrow W$ is said to be linear transformation (from $v \rightarrow W$), if:

	      $T(\alpha x + \beta y) = \alpha T(x) + \beta T(y)$

	      (for all $x,y \in W$ and $\alpha,\beta \in F$)

	      \textit{The above condition is said to be linearity property of transformation}

\end{enumerate}

\textbf{Note:} Geometrically, $T$ is a projection map and projecting point in $R^{3}$ to a point on $x$ axis.

\subsubsection{Examples}

\begin{qanda}

	\Q Show that the transformation $T:R^{3} \rightarrow R^{2}$ defined by $T(x,y,z) = (z,x+y)$ is linear.

	\A

	\begin{gather*}
		\text{Let }u,v \in R^{3}\\
		\implies u = (a_{1}, a_{2}, a_{3})\\
		\text{Let }\alpha,\beta \in F\\
		\text{To prove: T is linear, i.e. }T(\alpha u + \beta v)\\\\
		\text{\textbf{LHS:}}\\
		= T(\alpha u + \beta v)\\
		= T(\alpha(a_{1}, a_{2}, a_{3}) + \beta(b_{1}, b_{2}, b_{3}))\\
		= T(\alpha a_{1} + \beta b_{1}, \alpha a_{2} + \beta b_{2}, \alpha a_{3} + \beta b_{3})\\
		= [(\alpha a_{3} + \beta b_{3}), (\alpha a_{1} + \beta b_{1}) + (\alpha a_{2} + \beta b_{3})]\\
		\\
		\therefore T(x,y,z) = (z, x + y)\\
		\\
		((\alpha a_{3} + \beta b_{3}), \alpha (a_{1} + a_{2}) + \beta(b_{1} + b_{2})) \rightarrow 1\\
		\\
		\text{\textbf{RHS:}}\\
		= \alpha T(u) + \beta T (v)\\
		= \alpha T(a_{1}, a_{2}, a_{3}) + \beta T(b_{1}, b_{2}, b_{3})\\
		= \alpha(a_{3}, a_{1} + a_{2}) + \beta (b_{3}, b_{1} + b_{2})\\
		= (\alpha a_{3}, \alpha (a_{1}, a_{2})) + (\beta b_{3}, \beta (b_{1} + b_{2}))\\
		= ((\alpha a_{3} + \beta b_{3}), \alpha (a_{1} + a_{2}) + \beta (b_{1} + b_{2})) \rightarrow 2\\
		\\
		\text{From 1 \& 2}\\
		\therefore T \text{ is linear}
	\end{gather*}

	\Q Test the map $T:R\rightarrow R$ defined by $T(x) = x + 3 ~\forall~ x \in R$ is a linear transformation

	\A

	\begin{gather*}
		\text{Let }x,y \in R\\
		\text{To prove: $T$ is linear, i.e.\ to prove: $T(\alpha x + \beta y) = \alpha T(x) + \beta T(y)$}\\
		\\
		\text{\textbf{LHS:}}\\
		T(\alpha x + \beta y )\\
		= (\alpha x + \beta y) + 3 \rightarrow 1\\
		\\
		\text{\textbf{RHS:}}\\
		\alpha (T(x)) + \beta (T(y))\\
		= \alpha (x + 3) + \beta ( y + 3) \rightarrow 2\\
		\\
		\text{From $1$ and $2$: LHS $\ne$ RHS}\\
		\therefore \text{$T$ is not linear}
	\end{gather*}

	\Q Check whether the transformation $T:R^{2} \rightarrow R^{2}$ defined by $T(x,y) = (\sin x, y)$ is linear or not

	\A

	\begin{gather*}
		\text{Let $x,y \in R^{2}$}\\
		\text{Let $x = (a_{1}, a_{2})$ \& $y = (b_{1}, b_{2})$}\\
		\text{Let $\alpha, \beta \in F$}\\
		\text{To prove: T is linear, i.e. $T(\alpha x + \beta y) = \alpha T(x) + \beta T(y)$}\\
		\\
		\text{\textbf{LHS:}}\\
		T(\alpha x + \beta y)\\
		= T(\alpha (a_{1}, a_{2}) + \beta(b_{1} + b_{2}))\\
		= T(\alpha a_{1} + \beta b_{1}, \alpha a_{2} + \beta b_{2})\\
		= (\sin (\alpha a_{1} + \beta b_{1}), \alpha a_{2} \beta b_{2}) \rightarrow 1\\
		\\
		\text{\textbf{RHS:}}\\
		\alpha T(x) + \beta T(y)\\
		= \alpha T(a_{1}, a_{2}) + \beta T(b_{1} b_{2})\\
		= \alpha (\sin a_{1}, a_{2}) + \beta (\sin b_{1}, b_{2})\\
		= (\alpha \sin a_{1} + \beta \sin b_{1}, \alpha a_{2} + \beta b_{2})\\
	\end{gather*}

	\Q Show that $T:R^{2} \rightarrow R^{2}$ defined by $T(a_{1}, a_{2}) = (2 a_{1} + a_{2}, a_{1})$ is linear

	\A

	\begin{gather*}
		\text{Let }x,y \in R^{2}\\
		x = (u_{1}, u_{2})\\
		y = (v_{1}, v_{2})
		\\
		\text{To check whether $T$ is linear or not:}\\
		T(\alpha x + \beta y) = \alpha T(x) + \beta T(y)\\
		\\
		\text{\textbf{LHS:}}\\
		T(\alpha x, \beta y ) = T(\alpha(u_{1}, u_{2}) + \beta (v_{1}, v_{2}))\\
		= T(\alpha u_{1} + \beta v_{1}, \alpha u_{2} + \beta u_{2})\\
		= (2 \alpha u_{1} + 2 \beta v_{1}, 2 \alpha u_{2} + 2 \beta v_{2}, \alpha u_{1} + \beta v_{1} )\\
		\\
		\text{\textbf{RHS:}}\\
		\alpha T(x) + \beta T (y) = \alpha T(u_{1}, u_{2}) = \alpha T(u_{1}, u_{2}) + \beta T(v_{1}, v_{2})\\
		\alpha (2u_{1} + 2u_{2}, u_{1}) + \beta (2v_{1} + 2v_{2}, v_{1})\\
		(\alpha 2 u_{1} + \alpha 2 u_{2}, \beta 2 v_{1} + \beta 2 v_{2}, \alpha u_{1} + \beta v_{1})\\
		\\
		\therefore \text{It is linear}
	\end{gather*}

	\Q Show that $T:R^{2} \rightarrow R^{2}$ defined by $T(x,y) = (x, -y) ~\forall~ x,y \in R^{2}$ is linear

	\A
	\begin{gather*}
		\text{Given:} \\
		T:R^{2} \rightarrow R^{2} \text{ defined by } T(x,y) = (x,-y)\\
		\text{To prove: $T$ is linear ($T(\alpha x + \beta y) = \alpha T(x) + \beta T (y)$)}\\
		x,y \in R^{2}~\&~\alpha, \beta \in F\\
		\implies x = (a_{1}, a_{2}) ~\&~ y = (b_{1}, b_{2})\\
		\\
		\text{\textbf{LHS:}}\\
		T(\alpha x + \beta y) = T(\alpha(a_{1}, a_{2}) + \beta(b_{1}, b_{2}))\\
		= T(\alpha a_{1} + \beta b_{1}, \alpha a_{2} + \beta b_{2})\\
		= (\alpha a_{1} + \beta b_{1}, -\alpha a_{2} - \beta b_{2})\\
		\\
		\text{\textbf{RHS:}}\\
		\alpha T(x) + \beta T(y) = \alpha T(a_{1}, a_{2}) + \beta T( b_{1}, b_{2})\\
		= \alpha(a_{1}, -a_{1}) + \beta (b_{1}, -b_{2})\\
		= (\alpha a_{1} + \beta b_{1}, -\alpha a_{2} - \beta b_{2}) \\
		\\
		\text{LHS $=$ RHS}\\
		\text{Equation is linear}
	\end{gather*}


	\Q Show that $T: R^{3} \rightarrow R^{3}$ defined by $T(x,y,z) = (x,0,0) ~\forall~ x,y,z \in R^{3}$ is linear

	\A
	\begin{gather*}
		\text{Given } T = R^{3} \rightarrow R^{3}\\
		T(x,y,z) = (x,0,0) \rightarrow 1\\
		\text{To prove, $T$ is linear. $T(\alpha x + \beta y = \alpha T(u) + \beta T(y))$}\\
		\\
		\text{Now: } u,v \in R^{3} ~\&~ \alpha, \beta \in F\\
		\implies u = (a_{1}, a_{2}, a_{3}) ~\&~ v = (b_{1}, b_{2}, b_{3})\\
		\\
		\text{\textbf{LHS:}}\\
		T(\alpha u + \beta v) = T(\alpha (a_{1}, a_{2}, a_{3}) + \beta(b_{1}, b_{2}, b_{3}))\\
		= T(\alpha a_{1} + \beta b_{1}, \alpha a_{2} + \beta b_{2}, \alpha a_{3} + \beta b_{3})\\
		= \alpha a_{1} + \beta b_{1} \text{ (from 1)}\\
		\\
		\text{\textbf{RHS:}}\\
		\alpha T(u) + \beta T(v) = \alpha T(a_{1}, a_{2}, a_{3}) + \beta T(b_{1}, b_{2}, b_{3})\\
		= \alpha (a_{1}) + \beta (b_{1}) \text{ (from 1)}\\
		\\
		\text{LHS $=$ RHS}\\
		\text{Equation is linear}
	\end{gather*}
\end{qanda}

\newpage

\subsection{Special Linear Transformations}

\begin{definition}[Zero Transformation/Zero Operator/Trivial Linear Transformation]
	Let $V$ and $W$ be two vector spaces over $F$.
	A map $T:V\rightarrow W$ defined by $T(v) = 0 ~\forall~ v \in V$ is called zero transformation.

	\textbf{Note: } $0(x) = 0 ~\forall~ x \in V$
\end{definition}

\begin{definition}[Identity Transformation/Operator]
	Let $V(F)$ be a vector space over a field $F$.
	A map $T:V\rightarrow V$ defined by $T(v) = v ~\forall~ v \in V$ is called identity transformation.

	\textbf{Note: }$I(x) = x ~\forall~ x \in V$
\end{definition}

\begin{theorem}
	Let $V$ and $W$ be vector spaces over $F$ and $T:V\rightarrow W$ will be a linear transformation
	$\implies T(\alpha u + \beta v) = \alpha T(u) + \beta T(v)$

	\\
	Prove that:

	\begin{enumerate}
		\item $T(0) = 0$
		      \begin{gather*}
			      \text{Put } \alpha = \beta = 0\\
			      T(0\times u + 0\times v) = 0\times T(u) + 0\times T(v)\\
			      = 0
		      \end{gather*}
		\item $T(-u) = - T(u)$
		      \begin{gather*}
			      \text{Put } \alpha = -1, \beta = 0\\
			      T(-1\times u + 0\times v) = -1 T(u) + 0\times T(v)\\
			      = -T(u)
		      \end{gather*}
		\item $T(u-v) = T(u) - T(v)$
		      \begin{gather*}
			      \text{Put } \alpha = 1, \beta = -1\\
			      T(1\times u - 1\times v) = 1\times T(u) - 1\times T(v)\\
			      = T(u) - T(v)
		      \end{gather*}
	\end{enumerate}
\end{theorem}


\subsubsection{Examples}


\begin{qanda}
	\Q Show that $T:R^{3} \rightarrow R_{2}$ defined by $T(x,y,z) = (|x|,y+z)$ is not linear.


  \A 

  \begin{gather*}
    \text{Given: }\\
    T: R^{3} \rightarrow R^{2}\\
    T(x,y,z) = (|x|, y + z)\\
    \\
    \text{Prove: $T(\alpha x + \beta y \ne \alpha T(x) + \beta T(y))$}\\
    \text{Let } u,v \in R^{3} ~\&~ \alpha, \beta \in F\\
    \implies u = (a_{1}, a_{2}, a_{3}), v = (b_{1}, b_{2}, b_{3})\\
    \\
    \text{\textbf{LHS:}}\\
    T(\alpha u + \beta v) = T(\alpha(a_{2}, b_{2}, c_{2}) + \beta (b_{1}, b_{2}, b_{3}))\\
    =T(\alpha a_{1} + \beta b_{1}, \alpha a_{2} + \beta b_{2}, \alpha a_{3} + \beta b_{3})\\
    = T(|\alpha a_{1} + \beta b_{1}|, (\alpha a_{2} + \beta b_{2}) + (\alpha a_{3} + \beta b_{3}))\\
    \\
    \text{\textbf{RHS}}\\
    \alpha T(x) + \beta T(y) = \alpha T(a_{1}, a_{2}, a_{3}) + \beta T(b_{1}, b_{2}, b_{3})\\
  = \alpha(|a_{1}| , a_{2} + a_{3}) + \beta (|b_{1}|, b_{2} + b_{3})\\
  = (\alpha | a_{1}| + \beta |b_{1}|, \alpha a_{2} + \beta b_{3} + \alpha a_{3} + \beta b_{3})\\
  \\
  \text{LHS $\ne$ RHS}\\
  \therefore \text{$T$ is not linear } ( \because |x + y| \ne |x| + |y| )
  \end{gather*}

  \Q Define $T:P_{n}(R) \rightarrow P_{n-1}(R)$ by $T(f(x)) = \frac{d}{dx}{f(x)} = f'(x)$
  Prove that $T$ is linear.

  \A

  \begin{gather*}
    \text{Given:}\\
    T:P_{n}(R) \rightarrow P_{n-1}(R)\\
    T(f{x}) = \frac{d}{dx}{f(x)} = f'(x)\\
    \text{Let } f(x), g(x) \in P_{n}(R) ~\&~ \alpha, \beta \in F\\
    \text{To Prove: } T(\alpha f(x) + \beta g(x)) = \alpha T(f(x)) + \beta T(g(x))\\
    \\
    \text{\textbf{LHS:}}\\
    T(\alpha(f(x)) + \beta (g(x))) = \alpha f'(x) + \beta g'(x)\\
    = \alpha T(f(x)) + \beta T(g(x))\\
    = \text{\textbf{RHS}}\\
    \\
    \therefore \text{$T$ is linear}
  \end{gather*}
\end{qanda}

\newpage

\section{Null Spaces and Ranges}

\subsection{Null Space}

\begin{definition}[Null Space]
  Let $V$ and $W$ be vector spaces over $F$ and $T:V\rightarrow W$ be a linear transformation.
  The null space (kernel) of $T$ denoted be $N(T)$ or Ker$(T)$ is a set of all vectors $x$ is $V$
  such that $T(x) = 0$, i.e. $N(T) = \{x \in C:T(x) = 0 \}$.
\end{definition}

\subsection{Range}

\begin{definition}[Range/Image]
  Let $V$ and $W$ be vector spaces over $F$ and $T:V\rightarrow W$ be a linear transformation.
  The range (image) of $T$ denoted by $R(T)$ or $im(T)$ is a subset of $W$ consisting of the images
  of all vectors $x$ in $V$.
  \[
    R(T) = \{ T(x): x \in V\}
  \]
\end{definition}

\subsection*{Examples}

\begin{qanda}
  
  \Q Let $T: R^{3} \rightarrow R^{2}$ be a linear transformation defined by $T(a_{1}, a_{2}, a_{3}) = (a_{1} - a_{2}, a_{3})$.
  Find $N(T)$ and $R(T)$

  \A

  \begin{gather*}
    \text{To find: } N(T)\\
    N(T) = \{x \in V: T(x) = 0 \}\\
    \text{Let $x \in N(T)$. Here $x = (a_{1}, a_{2}, a_{3}) \in R^{3}$}\\
    \implies T(x) - 0\\
    \implies T(a_{1}, a_{2}, a_{3}) = 0\\
    \implies (a_{1} - a_{2}, a_{3}) = (0,0)\\
    \implies a_{1} = a_{2}, a_{3} = 0\\
    \therefore x = (a_{1}, a_{1}, 0)\\
    N(T) = \{(a_{1}, a_{1}, 0) \in R^{3}: T(a_{1}, a_{1}, 0) = 0 \}\\
    \\
    \text{To find $R(T)$}\\
    R(T) = \{T(x): x \in V \} \subseteq W\\
    = \{T(a_{1}, a_{2}, a_{3}) (a_{1}, a_{2}, a_{3}) \in R^{3}\}\\
    = \{(a_{1} - a_{2}, a_{3}) (a_{1}, a_{2}, a_{3}) \in R^{3}\}
    \in R^{2}\\
    \text{This is the range $R(T)$}
  \end{gather*}


  \Q Let $V$ and $W$ be vector spaces over $F$ and $T:V\rightarrow W$ be a linear transformation.
  Prove that $N(T)$ and $R(T)$ are subspaces of $V$ and $W$ respectively.

  \A

  \begin{gather*}
    \text{Given: } T:V\rightarrow W \text{ is a linear transformation}\\
    \text{To prove $N(T)$ is a subspace of $V$}\\
    N(T) = \{x \in V:T(x) = 0\} \subseteq V\\
    \\
    T(0) = 0\\
    0 \in V ~\&~ T(0) = 0\\
    0 \in N(T)\\
    \implies N(T) \text{ is non empty}\\
    \\
    \text{Let }x,y \in N(T) ~\&~ \alpha, \beta \in F\\
    \implies T(x) = 0 ~\&~ T(y) = 0\\
    \\
    T(\alpha x + \beta y) = \alpha T(x) + \beta T(y)\\
    = \alpha T(0) + \beta T(0)\\
    = 0\\
    \implies \alpha x + \beta y \in N(T)\\
    \therefore N(T) \text{ is a subspace of $V$}\\
  \end{gather*}

  \begin{gather*}
    \text{To prove: $R(T)$ is a subspace of $W$}\\
    R(T) = \{T(x) = x \in V\} \subseteq W\\
    \text{As $T$ is linear}\\
    T(0v) = 0w\\
    x \in V\\
    T(x) \in W ~\&~ T(0) = 0\\
  \therefore R(T) \text{ is non-empty}\\
    \text{Let }x,y \in R(T) ~\&~ \alpha, \beta \in F \\
    \\
    \alpha x + \beta y \in R(T)\\
    x,y \in R(T)\\
    \implies \text{There exists $u,v \in V$ such that $T(u) = x ~\&~ T(u) = y$}\\
    \text{$T$ is a linear and $u,v \in V$}\\
    \\
    \therefore T(\alpha u + \beta v) = \alpha T(u) + \beta T(v) = \alpha x + \beta y\\ 
    \\
    \implies \alpha x + \beta y \text{ is the image of $\alpha u + \beta y$}\\
    \implies \alpha x + \beta y \in R(T)\\
    \implies R(T) \text{ is a subspace of $W$}
  \end{gather*}

\end{qanda}


\subsection{Rank and Nullity of Linear Transformation}


\begin{definition}[Rank of $T$]
  
Let $V$ and $W$ be vector spaces over $F$ and $T:V\rightarrow V$ be a linear Transformation.
If $V$ is finite dimensional, the dimension of range of $T$ is called \textit{Rank of $T$} denoted by rank$(T)$
  \[
  \text{rank$(T)$ = dim$(R(T))$} = n(R(T)) = n(\text{Basis of $R(T)$})
\]
\end{definition}


\begin{definition}[Nullity of $T$]
  
  Let $V$ and $W$ be vector spaces over $F$ and $T:V\rightarrow V$ be a linear Transformation.
  If $V$ is finite dimensional, the dimension of  null space of $T$ is called \textit{nullity of $T$} denoted by nullity$(T)$.
  \[
    \text{nullity$(T)$ = dim$(N(T))$ = $n$ (Basis of $N(T)$)}
  \]
\end{definition}


\begin{definition}[Span$(T)$]
  Let $V$ and $W$ be vector spaces over $F$ and $T:V \rightarrow W$ be a linear transformaion.
  If $B = \{e_{1}, e_{2}, \ldots, e_{n}\}$ are basis of $V$, then $R(T)$ = Span$(T)$.
  \[
    \text{Span$T(B)$ = Span}(T(v_{1}), T(v_{2}), \ldots, T(v_{n}))
  \]
\end{definition}


\textbf{Note:} Images of basis of vector space $V$ generates $R(T)$

\begin{theorem}[Dimension Theorem]
  Let $V$ and $W$ be a vector space of $F$ and let $T:v\rightarrow W$ be a linear transformation.
  $V$ is a finite dimensional then:
  \[
    \text{rank}(T) + \text{nullity}(T) = \text{dim}(V)
  \]
\end{theorem}

\newpage

\section{Injective and Surjective Linear Transformation}

\begin{definition}[Injective Linear Transformation]
  A linear transformation $T:V \rightarrow W$ is said to be injective or one-to-one lineat transformation
  if $T(x) = T(y)$, then $x = y$.
\end{definition}


\begin{definition}[Surjective Linear Transformation]
  A linear transformation $T:V\rightarrow W$ is said to be surjective or onto if every element $v in W$,
  there exists $V \in V$ such that $T(u) = v$
\end{definition}

\textbf{Note:}

\begin{enumerate}
  \item If $T$ is injective, then every element in $V$ has a unique image in $W$,
    or every element in the range of $T$ corresponds to exactly one element of its domain.
  \item If $T$ is surjective, then every element in $W$ has a pre-image in $V$.
\end{enumerate}

\begin{definition}[Bijective linear transformation]
  A linear transformation $T:V \rightarrow W$ is said to be bijective if its both one-to-one and onto.
\end{definition}

\subsection{Examples}

\begin{qanda}
  
  \Q If $T:V\rightarrow W$ is a linear transformation, then prove that $T$ is one-to-one $\iff N(T) = \{0\}$

  \A

  \begin{gather*}
    \text{Given: } T:V\rightarrow W \text{ is a lienar transformation}\\
    \text{Assume $T$ is one-to-one}\\
    \text{To prove: $N(T)$ = \{0\}}\\
    \text{Let $X\in N(T)$}\\
    \implies T(x) = 0\\
    \text{As $T$ is linear, $T(\alpha x + \beta y) = \alpha T(x) + \beta T(y)$ and $T(0) = 0$}\\
    \text{Put }\alpha = \beta = 0\\
    \implies \alpha T(x) + \beta T(y) = 0 \times 0 = 0\\
    T(x) = T(y)\\
    \implies x = y\\
    \implies x = 0 \text{ as $T$ is one-to-one}\\
    \implies N(T) = \{0\}
  \end{gather*}

  \begin{gather*}
    \text{\textbf{Part 2 (converse):}}\\
    \text{Assume: $N(T) = \{0\}$}\\
    \text{To prove: $T$ is one to one}\\
    \text{Assume: $T(x) = T(y)$}\\
    \text{Claim: $x = y$}\\
    T(x) - T(y) = 0\\
    \implies T(x-y) = 0\\
    (\because T \text{ is linear: } T(x-y) = T(x) - T(y))\\
    \implies x-y \in N(T)\\
    \implies x-y = 0\\
    \implies x = y\\
    \therefore \text{$T$ is one-to-one}
  \end{gather*}


  \Q Consider $T:R^{3} \rightarrow R^{2}$ defined by $T(a_{1}, a_{2}, a_{3}) = (a_{1} - a_{2}, 2 a_{3})$.
  Verify whether $T$ is one-to-one.

  \A 


  \begin{gather*}
    \text{Given: } T:R^{3} \rightarrow R^{2}\\
    T(a_{1}, a_{2}, a_{3}) = (a_{1} - a_{2}, 2 a_{3})\\
    % \text{\textit{To check $T$ is one-to-one, it is enough to verify if $N(T) = \{0\}$ or $N(T) = \{x \in V: T(x) = 0\}$}}\\
    \text{Let: $x \in N(T)$ where $x = (a_{1}, a_{2}, a_{3}) \in R^{3}$}\\
    \implies T(x) = 0\\
    \implies T(a_{1}, a_{2}, a_{3}) = 0\\
    \implies (a_{1} - a_{2}, 2 a_{3}) = (0,0)\\
    \implies a_{1} - a_{2} = 0, 2 a_{3} = 0\\
    \implies a_{1} = a_{2}, a_{3} = 0\\
    \therefore x = (a_{1}, a_{1}, 0)\\
    N(T) = \{(a_{1}, a_{1},0) \in R^{3}: T(a_{1}, a_{1}, 0) = 0 \}\\
    \implies N(T) \ne \{0\}\\
    \implies T \text{ is not one-to-one}
  \end{gather*}

  \Q Let $T:R^{2} \rightarrow R^{2}$ be a linear transformation defined by $T(a_{1}, a_{2}) = (a_{1} + a_{2}, a_{1})$.
  Check whether $T$ is one-to-one.

  \A 

  \begin{gather*}
    T(x) = 0\\
    \implies T(a_{1}, a_{2}) 0\\
    \implies (a_{1} + a_{2}, a_{1}) = (0,0)\\
    \implies a_{1} + a_{2} = 0, a_{1} = 0\\
    \implies a_{1} = -a_{2}, a_{2} = 0\\
    \implies a_{2} = 0\\
    \therefore x = (0,0)\\
    N(T) = \{(0,0) \in R^{2}:T(0,0) = 0\}\\
    \implies N(T) = \{0\}\\
    \therefore \text{$T$ is one-to-one}
  \end{gather*}

  \Q $T:P_{n}(R) \rightarrow P_{n+1}(R)$ defined by $T(f(x)) = \int_{0}^{x}f(x)dx$.
  Verify if $T$ is linear. Is $T$ one-to-one and onto?

  \A

  \begin{gather*}
    \text{\textbf{Part 1: Verify if $T$ is linear or not}}\\
    \\
    \text{To verify $T(\alpha f(x) + \beta f(y)) = \alpha T(x) + \beta T(y)$}\\
    ~\forall~ f(x), g(x) \in P_{n} ~\&~ \alpha, \beta \in F\\
    T(\alpha f(x) + \beta g(x)) = \int_{0}^{x} (\alpha f(x) + \beta g(x)) dx\\
    = \alpha \int_{0}^{x}f(x)dx + \beta \int_{0}^{x}g(x) dx\\
    = \alpha T(f(x)) + \beta T(g(x))\\
    \therefore \text{$T$ is linear}
  \end{gather*}

  \begin{gather*}
    \text{\textbf{Part 2: Verify whether $T$ is one-to-one or not}}\\
  \text{\textit{Dimension theorem: } dim( $N(T)$ ) + dim( $R(T)$ ) = dim $V$ }\\
  \text{Basis of $P_{n}(R)$ is: }\{1, x, x^{2}, \ldots, x^{n}\}\\
  \text{The images of these vectors generate $P(T)$}\\
  T(f(x)) = \int_{0}^{x}f(x)dx\\
  \implies T(1) =  \int_{0}^{x} 1.dx = {(x)}_{0}^{x} = x\\
  T(x) = \int_{0}^{x}x.dx  = {(\frac{x^{2}}{2})}^{x}_{0} = \frac{x^{2}}{2}\\
  T(x^{2}) = \int_{0}^{x} x^{2} dx = {(\frac{x^{3}}{3})}^{3}_{0} = \frac{x^{3}}{3}\\
  \text{In general:}\\
  T(x^{n}) - \frac{x^{n + 1}}{n + 1}~(n \ne -1)\\
  \\
  \therefore R(T) = \text{Span}(x, \frac{x^{2}}{2} \frac{x^{3}}{3}, \ldots, \frac{x^{n+1}}{n+1})\\
  \text{Consider: }\alpha_{1}(x) + \alpha _{2}(\frac{x^{2}}{2}) + \cdots + \alpha_{n+1}(\frac{x^{n+}}{n+1}) = 0\\ 
  \implies \alpha_{1} = \alpha_{2} = \cdots = \alpha_{n+1} = 0\\
  \implies \text{Vectors are linearly independent}\\
  \text{These vectors form basis of $R(T)$}\\
  \therefore \text{Basis of $R(T)$} = \{x, \frac{x^{2}}{2}, \frac{x^{3}}{3}, \ldots, \frac{x^{n+1}}{n+1}\}\\
  \end{gather*}

  \begin{gather*}
    \text{\textbf{To verify whether $T$ is onto or not}}\\
    \text{Basis of $P_{n}(R)$ is } \{1, x, x^{2}, x^{3}, \ldots, x^{n}\}\\
   \text{Basis of $R(T)$} = \{x, \frac{x^{2}}{2}, \frac{x^{3}}{3}, \ldots, \frac{x^{n+1}}{n+1}\}\\
    \text{Basis of $P_{n+1}(R)$ is } \{1, x, x^{2}, x^{3}, \ldots, x^{n}, x^{n+1}\}\\
    \text{There is no pre-image for 1}\\
    \therefore \text{$T$ is not onto}
  \end{gather*}

\end{qanda}




\end{document}
