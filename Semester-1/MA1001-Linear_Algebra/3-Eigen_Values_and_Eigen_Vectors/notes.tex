\documentclass[english,course,fleqn]{lecture}

% Packages
\usepackage{geometry}
\usepackage{amssymb}
\usepackage{amsmath}
\usepackage{amsthm}

% Page Setting
\geometry{a4paper}

% Custom environments
\newenvironment{qanda}{\begin{enumerate}\setlength{\parindent}{0pt}}{\medskip\end{enumerate}}
\newcommand{\Q}{\bigskip\bfseries \item}
\newcommand{\A}{\par\textbf{A:} \normalfont}

% Headers
\title{Eigen Values and Eigen Vectors}
\subtitle{Unit 3}
\shorttitle{Eigen Values and Eigen Vectors}
% \subject{Subject of the Talk}
\author{Pranaov S}
% \email{Author's email}
\speaker{Dr.\ Venugopal}
% \spemail{Speaker's email}
\ccode{MA1001}
\date{19}{10}{2024}
% \dateend{}{}{}
% \flag{An extra line if you need it}
% \attn{Something to get reader's attention}
\morelink{https://github.com/pranaovs/college-notes}

\begin{document}

\newpage

Introduction (Pending)

\lecture[]{23}{10}{2024}

\section{Eigen Values and Vectors}

\begin{definition}[Eigen Values]
  Let $T$ be a linear operator on a vector space $V$.
  A non-zero vector $V \in V$ is called an eigen vector of $T$ is there exists a scalar $\lambda$ such that $T \times v = \lambda v$.

  This scalar $\lambda$ is called eigen value corresponding to eigen vector $v$.
\end{definition}

\begin{theorem}[]
  Let $T$ be the lienat operator on a finite dimensional vector space $B$ and $\beta$ is an ordered basis for $B$.
  Then, $\lambda$ is called an eign value of $T \iff \lambda$ is called eigen valie of its matrix $T (A = [T]_{\beta})$.
\end{theorem}

\subsection{Method to find eigen values andeigen vectors}

\begin{enumerate}
  \item \textbf{Form characteristic equation for the given matrix}

    i.e. $|A-\lambda I| = 0$ \margintext{\begin{gather*}
        A \times v = \lambda v\\
        \implies A\times v - \lambda v = 0\\
        \implies (A - \lambda I) v = 0\\
        \implies |A - \lambda I| = 0
    \end{gather*}}


  \item \textbf{Solve the characteristic equation}

    The solution of the equation are known as \textbf{eigen values of the matrix}.

    \begin{gather*}
      A = \begin{bmatrix}
        a_{11} & a_{12} \\
        a_{21} & a_{22} \\
      \end{bmatrix}\\
      A - \lambda I =  \begin{bmatrix}
        a_{11} & a_{12} \\
        a_{21} & a_{22} \\
        \end{bmatrix} - \lambda \begin{bmatrix}
        1 & 0 \\
        0 & 1 \\
      \end{bmatrix}\\
      \implies \begin{bmatrix}
        a_{11} - \lambda & a_{12} \\
        a_{21} & a_{22} - \lambda \\
      \end{bmatrix}
    \end{gather*}

    Characteristic equation is

    \begin{gather*}
      |A - \lambda I| = 0\\
      \begin{vmatrix}
        a_{11} - \lambda & a_{12} \\
        a_{21} & a_{22} - \lambda \\
      \end{vmatrix} = 0\\
      \lambda^{2} - (a_{11} + a_{12})\lambda + (a_{11}a_{12} - a_{21} a _{22}) = 0
    \end{gather*}

  \item \textbf{Corresponding to each eigen value $\lambda_{i}$ an eigen vector $v_{i}$ is obtained by solving the matrix equation $(A - \lambda_{i}I) v_{i} = 0$}
\end{enumerate}

\subsection{Method to form characteristic equation}


Let $A$ be a square matrix of order $n$.
\[
  A = \begin{vmatrix}
    a_{11} & a_{12} & a_{13} & \cdots & a_{1n} \\
    a_{21} & a_{22} & a_{23} & \cdots & a_{2n} \\
    a_{31} & a_{32} & a_{33} & \cdots & a_{3n} \\
    \vdots & \vdots & \vdots & \ddots & \vdots \\
    a_{m1} & a_{m2} & a_{m3} & \cdots & a_{mn} \\
  \end{vmatrix}
\]

Characteristic equation is $A - \lambda I = 0$
\[
  A = \begin{vmatrix}
    a_{11} - \lambda & a_{12} & a_{13} & \cdots & a_{1n} \\
    a_{21} & a_{22} - \lambda & a_{23} & \cdots & a_{2n} \\
    a_{31} & a_{32} & a_{33} - \lambda & \cdots & a_{3n} \\
    \vdots & \vdots & \vdots & \ddots & \vdots \\
    a_{m1} & a_{m2} & a_{m3} & \cdots & a_{mn} - \lambda \\
  \end{vmatrix} = 0
\]

Expanding: \[
  (-1)^{n} \lambda^{n} + K_{1}\lambda^{n-1} + K_{2}\lambda^{n-2} + \cdots + K_{n} = 0
\]
where $K_{i}$'s are expressible in terms of elements of $a_{ij}$.

\subsubsection*{Shortcuts}

\begin{enumerate}
  \item Let $A$ be a $2 \times 2$ matrix.
    \[
      A = \begin{bmatrix}
        a_{11} & a_{12} \\
        a_{21} & a_{22} \\
      \end{bmatrix}
    \]
    \[
      | A - \lambda I| = \lambda^{2} - S_{1}\lambda + S_{2} = 0
    \]
    \begin{gather*}
      S_{1} =  \text{Trace of matrix } A = a_{11} + a_{22}\\
      S_{2} = |A|
    \end{gather*}

  \item Let $A$ be a $3 \times 3$ matrix
    \[
      A = \begin{bmatrix}
        a_{11} & a_{12} & a_{13} \\
        a_{21} & a_{22} & a_{23} \\
        a_{31} & a_{32} & a_{33} \\
      \end{bmatrix}
    \]
    \[
      |A - \lambda I| = \lambda^{3} - S_{1}\lambda^{2} + S_{2} \lambda - S_{3}
    \]

    \begin{gather*}
      S_{1} = tr(A)\\
      S_{2} = \text{Sum of minors of leading diagonal elements}\\
      = \begin{vmatrix}
        a_{22} & a_{23} \\
        a_{32} & a_{33} \\
        \end{vmatrix} + \begin{vmatrix}
        a_{11} & a_{13} \\
        a_{31} & a_{33} \\
        \end{vmatrix} + \begin{vmatrix}
        a_{11} & a_{12} \\
        a_{21} & a_{22} \\
      \end{vmatrix}\\
      S_{3} = |A|
    \end{gather*}
\end{enumerate}

\subsubsection{Questions}

\begin{qanda}
  \Q Find eigen value and eigen vectors of $A = \begin{bsmallmatrix}
    5 & 4 \\
    1 & 2 \\
  \end{bsmallmatrix}$

  \A The chatacteristic equation of $A$ is:
  \begin{gather*}
    |A = \lambda I| = 0\\
    \implies \lambda^{2} + S_{1} \lambda + S_{2} = 0\\
    S_{1} = tr(A) = 7\\
    S_{2} = |A| = 6
    \\
    \therefore |A - \lambda I| = \lambda^{2} - 7 \lambda + 6 = 0\\
    \lambda_{1} = 6;\lambda_{2} = 1
  \end{gather*}

  To find eigen vectors:
  \begin{gather*}
    (A - \lambda_{i}I)v_{i} = 0\\
    \begin{bmatrix}
      5 - \lambda & 4 \\
      4 & 2 - \lambda \\
    \end{bmatrix} \begin{bmatrix}
      x_{1} \\
      x_{2} \\
    \end{bmatrix} = \begin{bmatrix}
      0 \\
      0 \\
    \end{bmatrix}\\
    \text{Case 1: } \lambda_{1} = 6\\
    \begin{bmatrix}
      5 - 6 & 4 \\
      4 & 2 - 6 \\
    \end{bmatrix} \begin{bmatrix}
      x_{1} \\
      x_{2} \\
    \end{bmatrix} = \begin{bmatrix}
      0 \\
      0 \\
    \end{bmatrix}\\
    = \begin{bmatrix}
      -1 & 4 \\
      4 & -4 \\
    \end{bmatrix} \begin{bmatrix}
      x_{1} \\
      x_{2} \\
    \end{bmatrix} = \begin{bmatrix}
      0 \\
      0 \\
    \end{bmatrix}\\
    \\
    -x_{1} + 4 x_{2} = 0\\
    x_{1} - 4 x_{2} = 0\\
    \implies x_{1} = 4 x_{2}\\
    \text{Let } x_{2} = 1; x_{1} = 4
  \end{gather*}

  Eigen vector is $v_{1} = \begin{bsmallmatrix}
    4 \\
    1 \\
  \end{bsmallmatrix}$

  $||| ~ v_{2} = \begin{bmatrix}
    -1 \\
    1 \\
  \end{bmatrix}$
\end{qanda}

\end{document}
